


% Header, overrides base

    % Make sure that the sphinx doc style knows who it inherits from.
    \def\sphinxdocclass{article}

    % Declare the document class
    \documentclass[letterpaper,10pt,english]{/usr/local/lib/python3.4/dist-packages/sphinx/texinputs/sphinxhowto}

    % Imports
    \usepackage[utf8]{inputenc}
    \DeclareUnicodeCharacter{00A0}{\\nobreakspace}
    \usepackage[T1]{fontenc}
    \usepackage{babel}
    \usepackage{times}
    \usepackage{import}
    \usepackage[Bjarne]{/usr/local/lib/python3.4/dist-packages/sphinx/texinputs/fncychap}
    \usepackage{longtable}
    \usepackage{/usr/local/lib/python3.4/dist-packages/sphinx/texinputs/sphinx}
    \usepackage{multirow}

    \usepackage{amsmath}
    \usepackage{amssymb}
    \usepackage{ucs}
    \usepackage{enumerate}

    % Used to make the Input/Output rules follow around the contents.
    \usepackage{needspace}

    % Pygments requirements
    \usepackage{fancyvrb}
    \usepackage{color}
    % ansi colors additions
    \definecolor{darkgreen}{rgb}{.12,.54,.11}
    \definecolor{lightgray}{gray}{.95}
    \definecolor{brown}{rgb}{0.54,0.27,0.07}
    \definecolor{purple}{rgb}{0.5,0.0,0.5}
    \definecolor{darkgray}{gray}{0.25}
    \definecolor{lightred}{rgb}{1.0,0.39,0.28}
    \definecolor{lightgreen}{rgb}{0.48,0.99,0.0}
    \definecolor{lightblue}{rgb}{0.53,0.81,0.92}
    \definecolor{lightpurple}{rgb}{0.87,0.63,0.87}
    \definecolor{lightcyan}{rgb}{0.5,1.0,0.83}

    % Needed to box output/input
    \usepackage{tikz}
        \usetikzlibrary{calc,arrows,shadows}
    \usepackage[framemethod=tikz]{mdframed}

    \usepackage{alltt}

    % Used to load and display graphics
    \usepackage{graphicx}
    \graphicspath{ {figs/} }
    \usepackage[Export]{adjustbox} % To resize

    % used so that images for notebooks which have spaces in the name can still be included
    \usepackage{grffile}


    % For formatting output while also word wrapping.
    \usepackage{listings}
    \lstset{breaklines=true}
    \lstset{basicstyle=\small\ttfamily}
    \def\smaller{\fontsize{9.5pt}{9.5pt}\selectfont}

    %Pygments definitions
    
\makeatletter
\def\PY@reset{\let\PY@it=\relax \let\PY@bf=\relax%
    \let\PY@ul=\relax \let\PY@tc=\relax%
    \let\PY@bc=\relax \let\PY@ff=\relax}
\def\PY@tok#1{\csname PY@tok@#1\endcsname}
\def\PY@toks#1+{\ifx\relax#1\empty\else%
    \PY@tok{#1}\expandafter\PY@toks\fi}
\def\PY@do#1{\PY@bc{\PY@tc{\PY@ul{%
    \PY@it{\PY@bf{\PY@ff{#1}}}}}}}
\def\PY#1#2{\PY@reset\PY@toks#1+\relax+\PY@do{#2}}

\expandafter\def\csname PY@tok@sx\endcsname{\def\PY@tc##1{\textcolor[rgb]{0.00,0.50,0.00}{##1}}}
\expandafter\def\csname PY@tok@s\endcsname{\def\PY@tc##1{\textcolor[rgb]{0.73,0.13,0.13}{##1}}}
\expandafter\def\csname PY@tok@cp\endcsname{\def\PY@tc##1{\textcolor[rgb]{0.74,0.48,0.00}{##1}}}
\expandafter\def\csname PY@tok@kd\endcsname{\let\PY@bf=\textbf\def\PY@tc##1{\textcolor[rgb]{0.00,0.50,0.00}{##1}}}
\expandafter\def\csname PY@tok@cs\endcsname{\let\PY@it=\textit\def\PY@tc##1{\textcolor[rgb]{0.25,0.50,0.50}{##1}}}
\expandafter\def\csname PY@tok@cm\endcsname{\let\PY@it=\textit\def\PY@tc##1{\textcolor[rgb]{0.25,0.50,0.50}{##1}}}
\expandafter\def\csname PY@tok@nb\endcsname{\def\PY@tc##1{\textcolor[rgb]{0.00,0.50,0.00}{##1}}}
\expandafter\def\csname PY@tok@se\endcsname{\let\PY@bf=\textbf\def\PY@tc##1{\textcolor[rgb]{0.73,0.40,0.13}{##1}}}
\expandafter\def\csname PY@tok@nv\endcsname{\def\PY@tc##1{\textcolor[rgb]{0.10,0.09,0.49}{##1}}}
\expandafter\def\csname PY@tok@sh\endcsname{\def\PY@tc##1{\textcolor[rgb]{0.73,0.13,0.13}{##1}}}
\expandafter\def\csname PY@tok@nt\endcsname{\let\PY@bf=\textbf\def\PY@tc##1{\textcolor[rgb]{0.00,0.50,0.00}{##1}}}
\expandafter\def\csname PY@tok@nd\endcsname{\def\PY@tc##1{\textcolor[rgb]{0.67,0.13,1.00}{##1}}}
\expandafter\def\csname PY@tok@ge\endcsname{\let\PY@it=\textit}
\expandafter\def\csname PY@tok@sd\endcsname{\let\PY@it=\textit\def\PY@tc##1{\textcolor[rgb]{0.73,0.13,0.13}{##1}}}
\expandafter\def\csname PY@tok@mf\endcsname{\def\PY@tc##1{\textcolor[rgb]{0.40,0.40,0.40}{##1}}}
\expandafter\def\csname PY@tok@il\endcsname{\def\PY@tc##1{\textcolor[rgb]{0.40,0.40,0.40}{##1}}}
\expandafter\def\csname PY@tok@go\endcsname{\def\PY@tc##1{\textcolor[rgb]{0.53,0.53,0.53}{##1}}}
\expandafter\def\csname PY@tok@sc\endcsname{\def\PY@tc##1{\textcolor[rgb]{0.73,0.13,0.13}{##1}}}
\expandafter\def\csname PY@tok@kc\endcsname{\let\PY@bf=\textbf\def\PY@tc##1{\textcolor[rgb]{0.00,0.50,0.00}{##1}}}
\expandafter\def\csname PY@tok@gd\endcsname{\def\PY@tc##1{\textcolor[rgb]{0.63,0.00,0.00}{##1}}}
\expandafter\def\csname PY@tok@kp\endcsname{\def\PY@tc##1{\textcolor[rgb]{0.00,0.50,0.00}{##1}}}
\expandafter\def\csname PY@tok@w\endcsname{\def\PY@tc##1{\textcolor[rgb]{0.73,0.73,0.73}{##1}}}
\expandafter\def\csname PY@tok@c1\endcsname{\let\PY@it=\textit\def\PY@tc##1{\textcolor[rgb]{0.25,0.50,0.50}{##1}}}
\expandafter\def\csname PY@tok@mo\endcsname{\def\PY@tc##1{\textcolor[rgb]{0.40,0.40,0.40}{##1}}}
\expandafter\def\csname PY@tok@s2\endcsname{\def\PY@tc##1{\textcolor[rgb]{0.73,0.13,0.13}{##1}}}
\expandafter\def\csname PY@tok@kn\endcsname{\let\PY@bf=\textbf\def\PY@tc##1{\textcolor[rgb]{0.00,0.50,0.00}{##1}}}
\expandafter\def\csname PY@tok@sb\endcsname{\def\PY@tc##1{\textcolor[rgb]{0.73,0.13,0.13}{##1}}}
\expandafter\def\csname PY@tok@gu\endcsname{\let\PY@bf=\textbf\def\PY@tc##1{\textcolor[rgb]{0.50,0.00,0.50}{##1}}}
\expandafter\def\csname PY@tok@na\endcsname{\def\PY@tc##1{\textcolor[rgb]{0.49,0.56,0.16}{##1}}}
\expandafter\def\csname PY@tok@gi\endcsname{\def\PY@tc##1{\textcolor[rgb]{0.00,0.63,0.00}{##1}}}
\expandafter\def\csname PY@tok@mh\endcsname{\def\PY@tc##1{\textcolor[rgb]{0.40,0.40,0.40}{##1}}}
\expandafter\def\csname PY@tok@mb\endcsname{\def\PY@tc##1{\textcolor[rgb]{0.40,0.40,0.40}{##1}}}
\expandafter\def\csname PY@tok@gh\endcsname{\let\PY@bf=\textbf\def\PY@tc##1{\textcolor[rgb]{0.00,0.00,0.50}{##1}}}
\expandafter\def\csname PY@tok@m\endcsname{\def\PY@tc##1{\textcolor[rgb]{0.40,0.40,0.40}{##1}}}
\expandafter\def\csname PY@tok@vi\endcsname{\def\PY@tc##1{\textcolor[rgb]{0.10,0.09,0.49}{##1}}}
\expandafter\def\csname PY@tok@ni\endcsname{\let\PY@bf=\textbf\def\PY@tc##1{\textcolor[rgb]{0.60,0.60,0.60}{##1}}}
\expandafter\def\csname PY@tok@nl\endcsname{\def\PY@tc##1{\textcolor[rgb]{0.63,0.63,0.00}{##1}}}
\expandafter\def\csname PY@tok@si\endcsname{\let\PY@bf=\textbf\def\PY@tc##1{\textcolor[rgb]{0.73,0.40,0.53}{##1}}}
\expandafter\def\csname PY@tok@c\endcsname{\let\PY@it=\textit\def\PY@tc##1{\textcolor[rgb]{0.25,0.50,0.50}{##1}}}
\expandafter\def\csname PY@tok@gr\endcsname{\def\PY@tc##1{\textcolor[rgb]{1.00,0.00,0.00}{##1}}}
\expandafter\def\csname PY@tok@k\endcsname{\let\PY@bf=\textbf\def\PY@tc##1{\textcolor[rgb]{0.00,0.50,0.00}{##1}}}
\expandafter\def\csname PY@tok@vg\endcsname{\def\PY@tc##1{\textcolor[rgb]{0.10,0.09,0.49}{##1}}}
\expandafter\def\csname PY@tok@kt\endcsname{\def\PY@tc##1{\textcolor[rgb]{0.69,0.00,0.25}{##1}}}
\expandafter\def\csname PY@tok@ow\endcsname{\let\PY@bf=\textbf\def\PY@tc##1{\textcolor[rgb]{0.67,0.13,1.00}{##1}}}
\expandafter\def\csname PY@tok@gs\endcsname{\let\PY@bf=\textbf}
\expandafter\def\csname PY@tok@nf\endcsname{\def\PY@tc##1{\textcolor[rgb]{0.00,0.00,1.00}{##1}}}
\expandafter\def\csname PY@tok@no\endcsname{\def\PY@tc##1{\textcolor[rgb]{0.53,0.00,0.00}{##1}}}
\expandafter\def\csname PY@tok@bp\endcsname{\def\PY@tc##1{\textcolor[rgb]{0.00,0.50,0.00}{##1}}}
\expandafter\def\csname PY@tok@o\endcsname{\def\PY@tc##1{\textcolor[rgb]{0.40,0.40,0.40}{##1}}}
\expandafter\def\csname PY@tok@nc\endcsname{\let\PY@bf=\textbf\def\PY@tc##1{\textcolor[rgb]{0.00,0.00,1.00}{##1}}}
\expandafter\def\csname PY@tok@gt\endcsname{\def\PY@tc##1{\textcolor[rgb]{0.00,0.27,0.87}{##1}}}
\expandafter\def\csname PY@tok@nn\endcsname{\let\PY@bf=\textbf\def\PY@tc##1{\textcolor[rgb]{0.00,0.00,1.00}{##1}}}
\expandafter\def\csname PY@tok@ss\endcsname{\def\PY@tc##1{\textcolor[rgb]{0.10,0.09,0.49}{##1}}}
\expandafter\def\csname PY@tok@sr\endcsname{\def\PY@tc##1{\textcolor[rgb]{0.73,0.40,0.53}{##1}}}
\expandafter\def\csname PY@tok@gp\endcsname{\let\PY@bf=\textbf\def\PY@tc##1{\textcolor[rgb]{0.00,0.00,0.50}{##1}}}
\expandafter\def\csname PY@tok@err\endcsname{\def\PY@bc##1{\setlength{\fboxsep}{0pt}\fcolorbox[rgb]{1.00,0.00,0.00}{1,1,1}{\strut ##1}}}
\expandafter\def\csname PY@tok@ne\endcsname{\let\PY@bf=\textbf\def\PY@tc##1{\textcolor[rgb]{0.82,0.25,0.23}{##1}}}
\expandafter\def\csname PY@tok@vc\endcsname{\def\PY@tc##1{\textcolor[rgb]{0.10,0.09,0.49}{##1}}}
\expandafter\def\csname PY@tok@mi\endcsname{\def\PY@tc##1{\textcolor[rgb]{0.40,0.40,0.40}{##1}}}
\expandafter\def\csname PY@tok@s1\endcsname{\def\PY@tc##1{\textcolor[rgb]{0.73,0.13,0.13}{##1}}}
\expandafter\def\csname PY@tok@kr\endcsname{\let\PY@bf=\textbf\def\PY@tc##1{\textcolor[rgb]{0.00,0.50,0.00}{##1}}}

\def\PYZbs{\char`\\}
\def\PYZus{\char`\_}
\def\PYZob{\char`\{}
\def\PYZcb{\char`\}}
\def\PYZca{\char`\^}
\def\PYZam{\char`\&}
\def\PYZlt{\char`\<}
\def\PYZgt{\char`\>}
\def\PYZsh{\char`\#}
\def\PYZpc{\char`\%}
\def\PYZdl{\char`\$}
\def\PYZhy{\char`\-}
\def\PYZsq{\char`\'}
\def\PYZdq{\char`\"}
\def\PYZti{\char`\~}
% for compatibility with earlier versions
\def\PYZat{@}
\def\PYZlb{[}
\def\PYZrb{]}
\makeatother


    %Set pygments styles if needed...
    
        \definecolor{nbframe-border}{rgb}{0.867,0.867,0.867}
        \definecolor{nbframe-bg}{rgb}{0.969,0.969,0.969}
        \definecolor{nbframe-in-prompt}{rgb}{0.0,0.0,0.502}
        \definecolor{nbframe-out-prompt}{rgb}{0.545,0.0,0.0}

        \newenvironment{ColorVerbatim}
        {\begin{mdframed}[%
            roundcorner=1.0pt, %
            backgroundcolor=nbframe-bg, %
            userdefinedwidth=1\linewidth, %
            leftmargin=0.1\linewidth, %
            innerleftmargin=0pt, %
            innerrightmargin=0pt, %
            linecolor=nbframe-border, %
            linewidth=1pt, %
            usetwoside=false, %
            everyline=true, %
            innerlinewidth=3pt, %
            innerlinecolor=nbframe-bg, %
            middlelinewidth=1pt, %
            middlelinecolor=nbframe-bg, %
            outerlinewidth=0.5pt, %
            outerlinecolor=nbframe-border, %
            needspace=0pt
        ]}
        {\end{mdframed}}
        
        \newenvironment{InvisibleVerbatim}
        {\begin{mdframed}[leftmargin=0.1\linewidth,innerleftmargin=3pt,innerrightmargin=3pt, userdefinedwidth=1\linewidth, linewidth=0pt, linecolor=white, usetwoside=false]}
        {\end{mdframed}}

        \renewenvironment{Verbatim}[1][\unskip]
        {\begin{alltt}\smaller}
        {\end{alltt}}
    
\usepackage{url}
\usepackage{hyperref}
\hypersetup{pdfborder = {0 0 0}}

%%% Header and Footer %%% 
\usepackage{lastpage}
\usepackage{fancyhdr}
%%% fancy pages for all non-title and chapter heading pages
\pagestyle{fancy}
%%% puts a line in the footer
\renewcommand{\footrulewidth}{0.4pt}
%%% adds a reference to the last page to give a running page count
\fancyfoot[R]{\thepage of \pageref{LastPage}} %requires lastpage
%%%% adds author name to bottom left
\fancyfoot[L]{\textit{Aaron Gonzales}}
%%% empty center footer
\fancyfoot[C]{}

%%% Adds class name to left header
\fancyhead[L]{\textit{Machine Learning}}
%%% Adds document name to right header
\fancyhead[R]{\textit{Project 2}}

    % Help prevent overflowing lines due to urls and other hard-to-break 
    % entities.  This doesn't catch everything...
    \sloppy

    % Document level variables
    \title{Machine Learning: Naive Bayes}
    \date{March 20, 2015}
    \release{}
    \author{Aaron Gonzales}
    \renewcommand{\releasename}{}

    % TODO: Add option for the user to specify a logo for his/her export.
    \newcommand{\sphinxlogo}{}

    % Make the index page of the document.
    \makeindex

    % Import sphinx document type specifics.
     


% Body

    % Start of the document
    \begin{document}

        
            \maketitle
        

        


        
        

    % Make sure that atleast 4 lines are below the HR
    \needspace{4\baselineskip}

    
        \vspace{6pt}
        \makebox[0.1\linewidth]{\smaller\hfill\tt\color{nbframe-in-prompt}In\hspace{4pt}{[}1{]}:\hspace{4pt}}\\*
        \vspace{-2.65\baselineskip}
        \begin{ColorVerbatim}
            \vspace{-0.7\baselineskip}
            \begin{Verbatim}[commandchars=\\\{\}]
\PY{o}{\PYZpc{}}\PY{k}{load\PYZus{}ext} \PY{n}{autoreload}
\PY{o}{\PYZpc{}}\PY{k}{autoreload} \PY{l+m+mi}{1}
\PY{o}{\PYZpc{}}\PY{k}{aimport} \PY{n}{bayes}
\PY{o}{\PYZpc{}}\PY{k}{aimport} \PY{n}{utils}
\PY{o}{\PYZpc{}}\PY{k}{pylab} \PY{n}{inline}
\PY{n}{pylab}\PY{o}{.}\PY{n}{rcParams}\PY{p}{[}\PY{l+s}{\PYZsq{}}\PY{l+s}{figure.figsize}\PY{l+s}{\PYZsq{}}\PY{p}{]} \PY{o}{=} \PY{p}{(}\PY{l+m+mf}{12.0}\PY{p}{,} \PY{l+m+mf}{12.0}\PY{p}{)}
\end{Verbatim}

            
                \vspace{-0.2\baselineskip}
            
        \end{ColorVerbatim}
    

    

        % If the first block is an image, minipage the image.  Else
        % request a certain amount of space for the input text.
        \needspace{4\baselineskip}
        
        

            % Add document contents.
            
                \begin{InvisibleVerbatim}
                \vspace{-0.5\baselineskip}
\begin{alltt}Populating the interactive namespace from numpy and matplotlib
\end{alltt}

            \end{InvisibleVerbatim}
            
        
    
\section{Naive Bayes Document
Classifier}\label{naive-bayes-document-classifier}

\subsection{Machine Learning Project 2
Report}\label{machine-learning-project-2-report}

\emph{Aaron Gonzales}

\emph{2015-03-18}

\begin{itemize}
\itemsep1pt\parskip0pt\parsep0pt
\item
  Note 1: This report was prepared using IPython Notebook. You will see
  code weaved in with the report and see output.
\item
  Note 2: I used the full twenty newsgroups dataset and not the
  pre-vectorized version in Matlab sparse format that we were given. The
  dataset comes with scikit-learn and is imported from there and is
  identical in content to the version processed by the set's maintainer,
  though I did tokenize it slightly differently by removing stopwords
  (they are not removed in the first set). Some code was used as
  examples from the Scikit community, though all of the classifer was
  implemented by me.
\end{itemize}\subsection{Question 1: Conditional
Independence}\label{question-1-conditional-independence}

The model described does not assume conditional independence and would
require \(O(2^n)\) estimations to be done, which is clearly out of the
question for a doument set.\subsection{Question 2, 3: Testing Accuracy and Prediction
Issues}\label{question-2-3-testing-accuracy-and-prediction-issues}

When doing any sort of classification task, it is crucial to evaluate
your performance as clearly as you can. To do this, we can run the
helper functions that give us this output after fitting the classifer.

This section will also explain how the flow of the program works.

\texttt{bayes.get\_newsgroups()} is a function that loads the newsgroups
dataset from scikit's built-in system. The returned objects hold the raw
data, and the data's class labels. The first document is printed below:

    % Make sure that atleast 4 lines are below the HR
    \needspace{4\baselineskip}

    
        \vspace{6pt}
        \makebox[0.1\linewidth]{\smaller\hfill\tt\color{nbframe-in-prompt}In\hspace{4pt}{[}2{]}:\hspace{4pt}}\\*
        \vspace{-2.65\baselineskip}
        \begin{ColorVerbatim}
            \vspace{-0.7\baselineskip}
            \begin{Verbatim}[commandchars=\\\{\}]
\PY{n}{twenty\PYZus{}train}\PY{p}{,} \PY{n}{twenty\PYZus{}test} \PY{o}{=} \PY{n}{bayes}\PY{o}{.}\PY{n}{get\PYZus{}newsgroups}\PY{p}{(}\PY{p}{)}
\PY{k}{print}\PY{p}{(}\PY{l+s}{\PYZsq{}}\PY{l+s}{\PYZhy{}\PYZhy{}\PYZhy{}\PYZhy{}\PYZhy{}\PYZhy{}\PYZhy{}\PYZhy{}\PYZhy{}\PYZhy{}\PYZhy{}}\PY{l+s}{\PYZsq{}}\PY{p}{)}
\PY{k}{print}\PY{p}{(}\PY{l+s}{\PYZsq{}}\PY{l+s}{Document 1 class: }\PY{l+s}{\PYZsq{}} \PY{o}{+} \PY{n+nb}{str}\PY{p}{(}\PY{n}{twenty\PYZus{}train}\PY{o}{.}\PY{n}{target\PYZus{}names}\PY{p}{[}\PY{n}{twenty\PYZus{}train}\PY{o}{.}\PY{n}{target}\PY{p}{[}\PY{l+m+mi}{0}\PY{p}{]}\PY{p}{]}\PY{p}{)}\PY{p}{)}
\PY{k}{print}\PY{p}{(}\PY{l+s}{\PYZsq{}}\PY{l+s}{\PYZhy{}\PYZhy{}\PYZhy{}\PYZhy{}\PYZhy{}\PYZhy{}\PYZhy{}\PYZhy{}\PYZhy{}\PYZhy{}\PYZhy{}}\PY{l+s}{\PYZsq{}}\PY{p}{)}
\PY{k}{print}\PY{p}{(}\PY{n}{twenty\PYZus{}train}\PY{o}{.}\PY{n}{data}\PY{p}{[}\PY{l+m+mi}{0}\PY{p}{]}\PY{p}{)}
\end{Verbatim}

            
                \vspace{-0.2\baselineskip}
            
        \end{ColorVerbatim}
    

    

        % If the first block is an image, minipage the image.  Else
        % request a certain amount of space for the input text.
        \needspace{4\baselineskip}
        
        

            % Add document contents.
            
                \begin{InvisibleVerbatim}
                \vspace{-0.5\baselineskip}
\begin{alltt}loading training set
loading testing set
-----------
Document 1 class: rec.sport.baseball
-----------
From: cubbie@garnet.berkeley.edu (                               )
Subject: Re: Cubs behind Marlins? How?
Article-I.D.: agate.1pt592\$f9a
Organization: University of California, Berkeley
Lines: 12
NNTP-Posting-Host: garnet.berkeley.edu


gajarsky@pilot.njin.net writes:

morgan and guzman will have era's 1 run higher than last year, and
 the cubs will be idiots and not pitch harkey as much as hibbard.
 castillo won't be good (i think he's a stud pitcher)

       This season so far, Morgan and Guzman helped to lead the Cubs
       at top in ERA, even better than THE rotation at Atlanta.
       Cubs ERA at 0.056 while Braves at 0.059. We know it is early
       in the season, we Cubs fans have learned how to enjoy the
       short triumph while it is still there.

\end{alltt}

            \end{InvisibleVerbatim}
            
        
    
I then vectorize the data into both a bag-of-words model and a TFIDF
model using scikits CountVectorizer() and TFiDFVectorizer().

    % Make sure that atleast 4 lines are below the HR
    \needspace{4\baselineskip}

    
        \vspace{6pt}
        \makebox[0.1\linewidth]{\smaller\hfill\tt\color{nbframe-in-prompt}In\hspace{4pt}{[}3{]}:\hspace{4pt}}\\*
        \vspace{-2.65\baselineskip}
        \begin{ColorVerbatim}
            \vspace{-0.7\baselineskip}
            \begin{Verbatim}[commandchars=\\\{\}]
\PY{n}{train\PYZus{}bow}\PY{p}{,} \PY{n}{test\PYZus{}bow}\PY{p}{,} \PY{n}{cv\PYZus{}bow} \PY{o}{=} \PY{n}{bayes}\PY{o}{.}\PY{n}{vectorize}\PY{p}{(}\PY{n}{twenty\PYZus{}train}\PY{o}{.}\PY{n}{data}\PY{p}{,} \PY{n}{twenty\PYZus{}test}\PY{o}{.}\PY{n}{data}\PY{p}{,} \PY{n}{model}\PY{o}{=}\PY{l+s}{\PYZsq{}}\PY{l+s}{bow}\PY{l+s}{\PYZsq{}}\PY{p}{)}
\PY{n}{train\PYZus{}tfidf}\PY{p}{,} \PY{n}{test\PYZus{}tfidf}\PY{p}{,} \PY{n}{cv\PYZus{}tfidf} \PY{o}{=} \PY{n}{bayes}\PY{o}{.}\PY{n}{vectorize}\PY{p}{(}\PY{n}{twenty\PYZus{}train}\PY{o}{.}\PY{n}{data}\PY{p}{,} \PY{n}{twenty\PYZus{}test}\PY{o}{.}\PY{n}{data}\PY{p}{,} \PY{n}{model}\PY{o}{=}\PY{l+s}{\PYZsq{}}\PY{l+s}{tfidf}\PY{l+s}{\PYZsq{}}\PY{p}{)}
\end{Verbatim}

            
                \vspace{-0.2\baselineskip}
            
        \end{ColorVerbatim}
    

    

        % If the first block is an image, minipage the image.  Else
        % request a certain amount of space for the input text.
        \needspace{4\baselineskip}
        
        

            % Add document contents.
            
                \begin{InvisibleVerbatim}
                \vspace{-0.5\baselineskip}
\begin{alltt}fitting training bow vector model
fitting test bow vector model
fitting training tfidf vector model
fitting test tfidf vector model
\end{alltt}

            \end{InvisibleVerbatim}
            
        
    
We then estimate the prior classes via MLE, the features give classes
via MAP (with dirchlet prior, though I will compare it with Laplacian
smoothing), and then predict values.

Each of these operations is highly vectorized using the Numpy package,
so where possible explict python for loops are avoided and operations
are cast as vector or matrix mutiplications or elementwise operations.
Each numpy array is structred to make it easy to perform these
operations and they are blazingly fast.

    % Make sure that atleast 4 lines are below the HR
    \needspace{4\baselineskip}

    
        \vspace{6pt}
        \makebox[0.1\linewidth]{\smaller\hfill\tt\color{nbframe-in-prompt}In\hspace{4pt}{[}4{]}:\hspace{4pt}}\\*
        \vspace{-2.65\baselineskip}
        \begin{ColorVerbatim}
            \vspace{-0.7\baselineskip}
            \begin{Verbatim}[commandchars=\\\{\}]
\PY{n}{class\PYZus{}priors} \PY{o}{=} \PY{n}{bayes}\PY{o}{.}\PY{n}{phat\PYZus{}class\PYZus{}est}\PY{p}{(}\PY{n}{twenty\PYZus{}train}\PY{o}{.}\PY{n}{target}\PY{p}{,} 
                                    \PY{n}{twenty\PYZus{}train}\PY{o}{.}\PY{n}{target\PYZus{}names}\PY{p}{,} 
                                    \PY{n}{debug}\PY{o}{=}\PY{n+nb+bp}{False}\PY{p}{)}

\PY{n}{phat\PYZus{}words} \PY{o}{=} \PY{n}{bayes}\PY{o}{.}\PY{n}{phat\PYZus{}word\PYZus{}est}\PY{p}{(}\PY{n}{train\PYZus{}bow}\PY{p}{,}
                                 \PY{n}{labels}\PY{o}{=}\PY{n}{twenty\PYZus{}train}\PY{o}{.}\PY{n}{target}
                                 \PY{c}{\PYZsh{}alpha=1}
                                 \PY{p}{)}

\PY{n}{predicted} \PY{o}{=} \PY{n}{bayes}\PY{o}{.}\PY{n}{predict}\PY{p}{(}\PY{n}{test\PYZus{}data}\PY{o}{=}\PY{n}{test\PYZus{}bow}\PY{p}{,}
              \PY{n}{test\PYZus{}labels}\PY{o}{=}\PY{n}{twenty\PYZus{}test}\PY{o}{.}\PY{n}{target}\PY{p}{,}
              \PY{n}{p\PYZus{}classes}\PY{o}{=}\PY{n}{class\PYZus{}priors}\PY{p}{,}
              \PY{n}{p\PYZus{}features}\PY{o}{=}\PY{n}{phat\PYZus{}words}
              \PY{p}{)}
\end{Verbatim}

            
                \vspace{-0.2\baselineskip}
            
        \end{ColorVerbatim}
    

    

        % If the first block is an image, minipage the image.  Else
        % request a certain amount of space for the input text.
        \needspace{4\baselineskip}
        
        

            % Add document contents.
            
                \begin{InvisibleVerbatim}
                \vspace{-0.5\baselineskip}
\begin{alltt}estimating params with Dirchlet Prior:
         vocabsize = 11314.000000
         beta = 0.000088
         alpha = 1.000088
         denominator = 1.000000
\end{alltt}

            \end{InvisibleVerbatim}
            
        
    
Now for the report:

    % Make sure that atleast 4 lines are below the HR
    \needspace{4\baselineskip}

    
        \vspace{6pt}
        \makebox[0.1\linewidth]{\smaller\hfill\tt\color{nbframe-in-prompt}In\hspace{4pt}{[}5{]}:\hspace{4pt}}\\*
        \vspace{-2.65\baselineskip}
        \begin{ColorVerbatim}
            \vspace{-0.7\baselineskip}
            \begin{Verbatim}[commandchars=\\\{\}]
\PY{n}{np}\PY{o}{.}\PY{n}{set\PYZus{}printoptions}\PY{p}{(}\PY{n}{precision}\PY{o}{=}\PY{l+m+mi}{4}\PY{p}{)}
\PY{n}{rep} \PY{o}{=} \PY{n}{utils}\PY{o}{.}\PY{n}{report}\PY{p}{(}\PY{n}{predicted}\PY{p}{,} \PY{n}{twenty\PYZus{}test}\PY{o}{.}\PY{n}{target\PYZus{}names}\PY{p}{,} \PY{n}{print\PYZus{}report}\PY{o}{=}\PY{n+nb+bp}{True}\PY{p}{,} \PY{n}{print\PYZus{}cm}\PY{o}{=}\PY{n+nb+bp}{True}\PY{p}{)}
\end{Verbatim}

            
                \vspace{-0.2\baselineskip}
            
        \end{ColorVerbatim}
    

    

        % If the first block is an image, minipage the image.  Else
        % request a certain amount of space for the input text.
        \needspace{4\baselineskip}
        
        

            % Add document contents.
            
                \begin{InvisibleVerbatim}
                \vspace{-0.5\baselineskip}
\begin{alltt}                          precision    recall  f1-score   support

             alt.atheism       0.84      0.78      0.81       319
           comp.graphics       0.64      0.66      0.65       389
 comp.os.ms-windows.misc       0.67      0.47      0.55       394
comp.sys.ibm.pc.hardware       0.62      0.65      0.64       392
   comp.sys.mac.hardware       0.84      0.64      0.72       385
          comp.windows.x       0.72      0.79      0.75       395
            misc.forsale       0.86      0.60      0.71       390
               rec.autos       0.87      0.83      0.85       396
         rec.motorcycles       0.93      0.92      0.92       398
      rec.sport.baseball       0.98      0.83      0.90       397
        rec.sport.hockey       0.89      0.97      0.93       399
               sci.crypt       0.65      0.94      0.77       396
         sci.electronics       0.74      0.57      0.64       393
                 sci.med       0.77      0.82      0.79       396
               sci.space       0.70      0.89      0.78       394
  soc.religion.christian       0.80      0.94      0.86       398
      talk.politics.guns       0.71      0.89      0.79       364
   talk.politics.mideast       0.85      0.96      0.90       376
      talk.politics.misc       0.65      0.65      0.65       310
      talk.religion.misc       0.85      0.45      0.59       251

             avg / total       0.78      0.77      0.76      7532

\end{alltt}

            \end{InvisibleVerbatim}
            
                \begin{InvisibleVerbatim}
                \vspace{-0.5\baselineskip}
    \begin{center}
    \includegraphics[max size={\textwidth}{\textheight}]{Naive_Bayes_files/Naive_Bayes_10_1.png}
    \par
    \end{center}
    
            \end{InvisibleVerbatim}
            
        
    
The classifier does well, hitting 77\% accuracy. It has a hard time with
two clusters of documents, the comp.* and the religious/political
groups. These groups have a great deal of overlap in subjects so this is
not too surprising.\subsection{Question 4: Priors and
Overfitting}\label{question-4-priors-and-overfitting}

The following blocks of code create our testing setup and fit the model
with many \(\beta\) values.

    % Make sure that atleast 4 lines are below the HR
    \needspace{4\baselineskip}

    
        \vspace{6pt}
        \makebox[0.1\linewidth]{\smaller\hfill\tt\color{nbframe-in-prompt}In\hspace{4pt}{[}5{]}:\hspace{4pt}}\\*
        \vspace{-2.65\baselineskip}
        \begin{ColorVerbatim}
            \vspace{-0.7\baselineskip}
            \begin{Verbatim}[commandchars=\\\{\}]

\end{Verbatim}

            
                \vspace{0.3\baselineskip}
            
        \end{ColorVerbatim}
    


    % Make sure that atleast 4 lines are below the HR
    \needspace{4\baselineskip}

    
        \vspace{6pt}
        \makebox[0.1\linewidth]{\smaller\hfill\tt\color{nbframe-in-prompt}In\hspace{4pt}{[}6{]}:\hspace{4pt}}\\*
        \vspace{-2.65\baselineskip}
        \begin{ColorVerbatim}
            \vspace{-0.7\baselineskip}
            \begin{Verbatim}[commandchars=\\\{\}]
\PY{n}{overfit\PYZus{}trials} \PY{o}{=} \PY{p}{[}\PY{l+m+mi}{1}\PY{o}{/}\PY{l+m+mi}{10}\PY{o}{*}\PY{o}{*}\PY{n}{x} \PY{k}{for} \PY{n}{x} \PY{o+ow}{in} \PY{n+nb}{reversed}\PY{p}{(}\PY{n+nb}{range}\PY{p}{(}\PY{l+m+mi}{10}\PY{p}{)}\PY{p}{)}\PY{p}{]}
\PY{k}{print}\PY{p}{(}\PY{n}{overfit\PYZus{}trials}\PY{p}{)}
\end{Verbatim}

            
                \vspace{-0.2\baselineskip}
            
        \end{ColorVerbatim}
    

    

        % If the first block is an image, minipage the image.  Else
        % request a certain amount of space for the input text.
        \needspace{4\baselineskip}
        
        

            % Add document contents.
            
                \begin{InvisibleVerbatim}
                \vspace{-0.5\baselineskip}
\begin{alltt}[1e-09, 1e-08, 1e-07, 1e-06, 1e-05, 0.0001, 0.001, 0.01, 0.1, 1.0]
\end{alltt}

            \end{InvisibleVerbatim}
            
        
    


    % Make sure that atleast 4 lines are below the HR
    \needspace{4\baselineskip}

    
        \vspace{6pt}
        \makebox[0.1\linewidth]{\smaller\hfill\tt\color{nbframe-in-prompt}In\hspace{4pt}{[}7{]}:\hspace{4pt}}\\*
        \vspace{-2.65\baselineskip}
        \begin{ColorVerbatim}
            \vspace{-0.7\baselineskip}
            \begin{Verbatim}[commandchars=\\\{\}]
\PY{o}{\PYZpc{}\PYZpc{}}\PY{k}{capture}
\PY{c}{\PYZsh{} capture magic command prevents printing by capturing stout}
\PY{n}{overfit\PYZus{}results} \PY{o}{=} \PY{p}{[}\PY{p}{]}
\PY{k}{for} \PY{n}{beta} \PY{o+ow}{in} \PY{n}{overfit\PYZus{}trials}\PY{p}{:}
    \PY{n}{phat\PYZus{}c}\PY{p}{,}\PY{n}{phat\PYZus{}w}\PY{p}{,} \PY{n}{pred}\PY{p}{,} \PY{n}{rep} \PY{o}{=} \PY{n}{bayes}\PY{o}{.}\PY{n}{run\PYZus{}model}\PY{p}{(}\PY{n}{twenty\PYZus{}train}\PY{p}{,} \PY{n}{twenty\PYZus{}test}\PY{p}{,}
                \PY{n}{beta}\PY{o}{=}\PY{n}{beta}\PY{p}{,}
                \PY{n}{bow}\PY{o}{=}\PY{p}{(}\PY{n}{train\PYZus{}bow}\PY{p}{,} \PY{n}{test\PYZus{}bow}\PY{p}{)}\PY{p}{,}
                \PY{n}{report}\PY{o}{=}\PY{n+nb+bp}{False}\PY{p}{)}
    \PY{n}{overfit\PYZus{}results}\PY{o}{.}\PY{n}{append}\PY{p}{(}\PY{n}{rep}\PY{p}{[}\PY{l+m+mi}{1}\PY{p}{]}\PY{p}{)}
\end{Verbatim}

            
                \vspace{-0.2\baselineskip}
            
        \end{ColorVerbatim}
    


    % Make sure that atleast 4 lines are below the HR
    \needspace{4\baselineskip}

    
        \vspace{6pt}
        \makebox[0.1\linewidth]{\smaller\hfill\tt\color{nbframe-in-prompt}In\hspace{4pt}{[}8{]}:\hspace{4pt}}\\*
        \vspace{-2.65\baselineskip}
        \begin{ColorVerbatim}
            \vspace{-0.7\baselineskip}
            \begin{Verbatim}[commandchars=\\\{\}]
\PY{n}{fit\PYZus{}results} \PY{o}{=} \PY{n}{np}\PY{o}{.}\PY{n}{array}\PY{p}{(}\PY{p}{(}\PY{n}{overfit\PYZus{}trials}\PY{p}{,} \PY{n}{overfit\PYZus{}results}\PY{p}{)}\PY{p}{)}
\PY{n}{pyplot}\PY{o}{.}\PY{n}{figure}\PY{p}{(}\PY{p}{)}
\PY{n}{plt}\PY{o}{.}\PY{n}{plot}\PY{p}{(}\PY{n}{fit\PYZus{}results}\PY{p}{[}\PY{l+m+mi}{0}\PY{p}{]}\PY{p}{,} \PY{n}{fit\PYZus{}results}\PY{p}{[}\PY{l+m+mi}{1}\PY{p}{]}\PY{p}{)}
\PY{c}{\PYZsh{} ax = fig.add\PYZus{}subplot(111)}
\PY{n}{plt}\PY{o}{.}\PY{n}{xscale}\PY{p}{(}\PY{l+s}{\PYZsq{}}\PY{l+s}{log}\PY{l+s}{\PYZsq{}}\PY{p}{)}
\PY{n}{plt}\PY{o}{.}\PY{n}{title}\PY{p}{(}\PY{l+s}{\PYZsq{}}\PY{l+s}{Effect of Beta choice on classifier accuracy}\PY{l+s}{\PYZsq{}}\PY{p}{)}
\PY{n}{plt}\PY{o}{.}\PY{n}{ylabel}\PY{p}{(}\PY{l+s}{\PYZsq{}}\PY{l+s}{classification accuracy}\PY{l+s}{\PYZsq{}}\PY{p}{)}
\PY{n}{plt}\PY{o}{.}\PY{n}{xlabel}\PY{p}{(}\PY{l+s}{\PYZsq{}}\PY{l+s}{beta}\PY{l+s}{\PYZsq{}}\PY{p}{)}
\PY{n}{plt}\PY{o}{.}\PY{n}{show}\PY{p}{(}\PY{p}{)}
\end{Verbatim}

            
                \vspace{-0.2\baselineskip}
            
        \end{ColorVerbatim}
    

    

        % If the first block is an image, minipage the image.  Else
        % request a certain amount of space for the input text.
        \needspace{4\baselineskip}
        
        

            % Add document contents.
            
                \begin{InvisibleVerbatim}
                \vspace{-0.5\baselineskip}
    \begin{center}
    \includegraphics[max size={\textwidth}{\textheight}]{Naive_Bayes_files/Naive_Bayes_16_0.png}
    \par
    \end{center}
    
            \end{InvisibleVerbatim}
            
        
    
As you can see, the choice of beta slightly changes the classification
score, with \(\beta = 0.1\) giving us our highest accuracy.

\[P(X_i|Y_k) = \frac{(count\ of\ X_i\ in\ Y_k)+(\alpha -1)}{(total\ words\ in Y_k)+((\alpha - 1)*(length\ of\ vocab\ list)))}\]

Where \[\alpha = 1 + \beta\]

tells us that \(\beta\) weights the amount of influence the size of the
vocabulary has over the estimated posterior probability.\subsection{Question 5,6: Important
Features}\label{question-56-important-features}

If we multiply the estimated word priors and the class priors, we can
see words that were important in our dataset. This is a sort-of indirect
estimate of \(P(C|W) = {P(W|C) * P(C)}\) This can allow us to pull the
top words per class or the top words overall by looking at words per
class.

    % Make sure that atleast 4 lines are below the HR
    \needspace{4\baselineskip}

    
        \vspace{6pt}
        \makebox[0.1\linewidth]{\smaller\hfill\tt\color{nbframe-in-prompt}In\hspace{4pt}{[}8{]}:\hspace{4pt}}\\*
        \vspace{-2.65\baselineskip}
        \begin{ColorVerbatim}
            \vspace{-0.7\baselineskip}
            \begin{Verbatim}[commandchars=\\\{\}]
\PY{c}{\PYZsh{} numpy will broadcast the smaller array along the larger array}
\PY{c}{\PYZsh{} and we transpose it to get it back}
\PY{c}{\PYZsh{} see http://wiki.scipy.org/EricsBroadcastingDoc}
\PY{n}{log\PYZus{}posteriors} \PY{o}{=}  \PY{n}{np}\PY{o}{.}\PY{n}{transpose}\PY{p}{(}\PY{n}{np}\PY{o}{.}\PY{n}{log2}\PY{p}{(}\PY{n}{phat\PYZus{}c}\PY{p}{)} \PY{o}{+} \PY{n}{np}\PY{o}{.}\PY{n}{log2}\PY{p}{(}\PY{n}{phat\PYZus{}w}\PY{o}{.}\PY{n}{T}\PY{p}{)}\PY{p}{)}
\PY{n}{best\PYZus{}words} \PY{o}{=} \PY{n}{utils}\PY{o}{.}\PY{n}{top\PYZus{}words}\PY{p}{(}\PY{n}{cv\PYZus{}bow}\PY{p}{,} \PY{n}{log\PYZus{}posteriors}\PY{p}{,}
                \PY{n}{twenty\PYZus{}test}\PY{o}{.}\PY{n}{target\PYZus{}names}\PY{p}{,}
                \PY{n}{n}\PY{o}{=}\PY{l+m+mi}{100}\PY{p}{,} \PY{n}{per\PYZus{}class}\PY{o}{=}\PY{n+nb+bp}{False}\PY{p}{,} \PY{n}{order}\PY{o}{=}\PY{l+s}{\PYZsq{}}\PY{l+s}{high}\PY{l+s}{\PYZsq{}}\PY{p}{)}
\PY{n}{worst\PYZus{}words} \PY{o}{=} \PY{n}{utils}\PY{o}{.}\PY{n}{top\PYZus{}words}\PY{p}{(}\PY{n}{cv\PYZus{}bow}\PY{p}{,} \PY{n}{log\PYZus{}posteriors}\PY{p}{,}
                \PY{n}{twenty\PYZus{}test}\PY{o}{.}\PY{n}{target\PYZus{}names}\PY{p}{,}
                \PY{n}{n}\PY{o}{=}\PY{l+m+mi}{100}\PY{p}{,} \PY{n}{per\PYZus{}class}\PY{o}{=}\PY{n+nb+bp}{False}\PY{p}{,} \PY{n}{order}\PY{o}{=}\PY{l+s}{\PYZsq{}}\PY{l+s}{low}\PY{l+s}{\PYZsq{}}\PY{p}{)}
\PY{n}{best} \PY{o}{=} \PY{p}{[}\PY{n}{utils}\PY{o}{.}\PY{n}{get\PYZus{}word\PYZus{}count}\PY{p}{(}\PY{n}{cv\PYZus{}bow}\PY{p}{,} \PY{n}{train\PYZus{}bow}\PY{p}{,} \PY{n}{word}\PY{p}{)} 
                        \PY{k}{for} \PY{n}{word} \PY{o+ow}{in} \PY{n}{best\PYZus{}words}\PY{p}{]}
\PY{n}{worst} \PY{o}{=} \PY{p}{[}\PY{n}{utils}\PY{o}{.}\PY{n}{get\PYZus{}word\PYZus{}count}\PY{p}{(}\PY{n}{cv\PYZus{}bow}\PY{p}{,} \PY{n}{train\PYZus{}bow}\PY{p}{,} \PY{n}{word}\PY{p}{)} 
                         \PY{k}{for} \PY{n}{word} \PY{o+ow}{in} \PY{n}{worst\PYZus{}words}\PY{p}{]}
\end{Verbatim}

            
                \vspace{-0.2\baselineskip}
            
        \end{ColorVerbatim}
    


    % Make sure that atleast 4 lines are below the HR
    \needspace{4\baselineskip}

    
        \vspace{6pt}
        \makebox[0.1\linewidth]{\smaller\hfill\tt\color{nbframe-in-prompt}In\hspace{4pt}{[}69{]}:\hspace{4pt}}\\*
        \vspace{-2.65\baselineskip}
        \begin{ColorVerbatim}
            \vspace{-0.7\baselineskip}
            \begin{Verbatim}[commandchars=\\\{\}]
\PY{k}{print}\PY{p}{(}\PY{n}{best}\PY{p}{)}
\end{Verbatim}

            
                \vspace{-0.2\baselineskip}
            
        \end{ColorVerbatim}
    

    

        % If the first block is an image, minipage the image.  Else
        % request a certain amount of space for the input text.
        \needspace{4\baselineskip}
        
        

            % Add document contents.
            
                \begin{InvisibleVerbatim}
                \vspace{-0.5\baselineskip}
\begin{alltt}[('controller', 439), ('launch', 361), ('baseball', 454), ('pitt',
567), ('games', 819), ('mr', 1215), ('turkey', 373), ('55', 551),
('information', 1983), ('faith', 582), ('think', 4578), ('25', 1435),
('president', 860), ('2tm', 354), ('card', 1164), ('entry', 456),
('wm', 396), ('escrow', 360), ('bible', 777), ('ide', 410),
('armenia', 387), ('cars', 473), ('christian', 938), ('10', 2462),
('christ', 567), ('believe', 2026), ('public', 1426), ('season', 624),
('cx', 443), ('widget', 409), ('program', 1689), ('use', 4183),
('christians', 680), ('mit', 999), ('nhl', 419), ('guns', 563),
('year', 2055), ('privacy', 450), ('giz', 433), ('church', 638),
('apple', 857), ('security', 663), ('keys', 611), ('server', 758),
('motif', 499), ('play', 879), ('dod', 543), ('bhj', 473), ('said',
2453), ('mac', 919), ('government', 1814), ('2di', 503), ('com',
12133), ('image', 805), ('3t', 521), ('1t', 528), ('75u', 529),
('file', 2005), ('ca', 3544), ('db', 649), ('sale', 758), ('bike',
569), ('edu', 21321), ('jews', 866), ('game', 1194), ('graphics',
945), ('0t', 586), ('00', 1534), ('armenians', 624), ('jesus', 1239),
('34u', 639), ('israeli', 685), ('drive', 1626), ('people', 5975),
('hockey', 660), ('0d', 684), ('armenian', 717), ('team', 1171),
('1d9', 705), ('turkish', 748), ('nasa', 1335), ('chip', 1119),
('scsi', 978), ('clipper', 793), ('window', 1126), ('pl', 856),
('car', 1311), ('encryption', 860), ('israel', 1014), ('gun', 1179),
('145', 996), ('a86', 1040), ('windows', 1953), ('b8f', 1156), ('g9v',
1203), ('key', 1674), ('god', 2994), ('space', 1856), ('max', 4690),
('ax', 62406)]
\end{alltt}

            \end{InvisibleVerbatim}
            
        
    
Notice how the words that come out all have moderate counts and tent to
be loaded - car, file, mac, government, jesus - likely high information
words. Contrast those with the lowest info words in the set:

    % Make sure that atleast 4 lines are below the HR
    \needspace{4\baselineskip}

    
        \vspace{6pt}
        \makebox[0.1\linewidth]{\smaller\hfill\tt\color{nbframe-in-prompt}In\hspace{4pt}{[}70{]}:\hspace{4pt}}\\*
        \vspace{-2.65\baselineskip}
        \begin{ColorVerbatim}
            \vspace{-0.7\baselineskip}
            \begin{Verbatim}[commandchars=\\\{\}]
\PY{k}{print}\PY{p}{(}\PY{n}{worst}\PY{p}{)}
\end{Verbatim}

            
                \vspace{-0.2\baselineskip}
            
        \end{ColorVerbatim}
    

    

        % If the first block is an image, minipage the image.  Else
        % request a certain amount of space for the input text.
        \needspace{4\baselineskip}
        
        

            % Add document contents.
            
                \begin{InvisibleVerbatim}
                \vspace{-0.5\baselineskip}
\begin{alltt}[('campaigning', 6), ('softly', 5), ('mobs', 5), ('runaway', 5),
('733', 7), ('aquired', 5), ('1576', 5), ('reload', 7),
('distinctive', 6), ('architectural', 5), ('superseded', 5),
('laurier', 5), ('mach1', 5), ('wilfrid', 5), ('maze', 5),
('positioned', 7), ('solves', 7), ('adherence', 6), ('retailers', 6),
('troublesome', 6), ('fiddle', 6), ('supermarket', 6),
('contaminating', 5), ('1937', 5), ('burner', 5), ('7400', 5),
('uncover', 5), ('ploy', 5), ('addendum', 5), ('illuminating', 5),
('originals', 9), ('misinformed', 6), ('fone', 6), ('behaves', 6),
('subscriptions', 6), ('scrabble', 5), ('norms', 5), ('94086', 5),
('conceptual', 5), ('disregarding', 5), ('discrepancies', 5),
('unpublished', 5), ('vers', 5), ('archaic', 5), ('pioneers', 5),
('centrally', 5), ('primer', 5), ('consoles', 5), ('forefront', 5),
('prom', 5), ('glanced', 6), ('verge', 6), ('mater', 6),
('criminally', 5), ('boosting', 5), ('becasue', 5), ('hodges', 5),
('589', 5), ('\_\_\_\_\_\_\_\_\_\_\_\_\_\_\_\_\_\_\_\_\_\_\_\_\_\_\_\_\_\_\_\_\_\_\_\_\_\_\_\_\_\_\_\_\_\_\_\_\_\_\_\_\_\_\_\_
\_\_\_\_\_\_\_\_\_\_\_\_\_\_\_\_\_\_\_', 7), ('gratification', 7), ('sirius', 7),
('messing', 7), ('glued', 6), ('whims', 6), ('consolation', 6),
('nuff', 6), ('thirdly', 5), ('723', 5), ('stature', 5), ('roaring',
5), ('binder', 5), ('segundo', 5), ('strayed', 5), ('structurally',
5), ('tcs', 5), ('acoustic', 5), ('utmost', 9), ('fruitless', 9),
('mate', 9), ('comming', 7), ('generalized', 7), ('visions', 7),
('waits', 7), ('spurious', 7), ('exhibiting', 6), ('extremists', 6),
('stirred', 6), ('equivalence', 6), ('speculating', 6), ('restricts',
6), ('occupies', 6), ('exert', 6), ('397', 6), ('satisfying', 6),
('communicated', 6), ('misread', 5), ('entrusted', 5), ('ecstatic',
5), ('excellently', 5), ('debunk', 5)]
\end{alltt}

            \end{InvisibleVerbatim}
            
        
    
That have very, very low counts in the data, are often mispelled.

\subsection{Question 7: Bias}\label{question-7-bias}

It seems as though the words may have been slightly biased toward
religious and political words. If new data doesn't talk much about those
things, we might not accurately classify that data.

\subsection{Laplace, TFIDF}\label{laplace-tfidf}

I also wanted to report accuracies for Laplacian smoothing and then
TFIDF models.

    % Make sure that atleast 4 lines are below the HR
    \needspace{4\baselineskip}

    
        \vspace{6pt}
        \makebox[0.1\linewidth]{\smaller\hfill\tt\color{nbframe-in-prompt}In\hspace{4pt}{[}79{]}:\hspace{4pt}}\\*
        \vspace{-2.65\baselineskip}
        \begin{ColorVerbatim}
            \vspace{-0.7\baselineskip}
            \begin{Verbatim}[commandchars=\\\{\}]
\PY{n}{phat\PYZus{}c}\PY{p}{,}\PY{n}{phat\PYZus{}w}\PY{p}{,} \PY{n}{pred}\PY{p}{,} \PY{n}{rep} \PY{o}{=} \PY{n}{bayes}\PY{o}{.}\PY{n}{run\PYZus{}model}\PY{p}{(}\PY{n}{twenty\PYZus{}train}\PY{p}{,} \PY{n}{twenty\PYZus{}test}\PY{p}{,}
                \PY{n}{laplacian}\PY{o}{=}\PY{p}{(}\PY{n+nb+bp}{True}\PY{p}{,}\PY{l+m+mi}{1}\PY{p}{)}\PY{p}{,}
                \PY{n}{bow}\PY{o}{=}\PY{p}{(}\PY{n}{train\PYZus{}bow}\PY{p}{,} \PY{n}{test\PYZus{}bow}\PY{p}{)}\PY{p}{,}
                \PY{n}{report}\PY{o}{=}\PY{n+nb+bp}{True}\PY{p}{)}
\end{Verbatim}

            
                \vspace{-0.2\baselineskip}
            
        \end{ColorVerbatim}
    

    

        % If the first block is an image, minipage the image.  Else
        % request a certain amount of space for the input text.
        \needspace{4\baselineskip}
        
        

            % Add document contents.
            
                \begin{InvisibleVerbatim}
                \vspace{-0.5\baselineskip}
\begin{alltt}estimating class priors
estimating word priors
estimating params with Laplacian Smoothing:
         alpha = 1
         denominator = 20
predicting
                          precision    recall  f1-score   support

             alt.atheism       0.81      0.70      0.75       319
           comp.graphics       0.76      0.68      0.71       389
 comp.os.ms-windows.misc       0.77      0.49      0.60       394
comp.sys.ibm.pc.hardware       0.67      0.68      0.68       392
   comp.sys.mac.hardware       0.89      0.64      0.74       385
          comp.windows.x       0.64      0.87      0.74       395
            misc.forsale       0.95      0.59      0.73       390
               rec.autos       0.90      0.83      0.86       396
         rec.motorcycles       0.98      0.89      0.93       398
      rec.sport.baseball       0.98      0.83      0.90       397
        rec.sport.hockey       0.86      0.98      0.91       399
               sci.crypt       0.56      0.97      0.71       396
         sci.electronics       0.81      0.50      0.62       393
                 sci.med       0.84      0.80      0.82       396
               sci.space       0.72      0.93      0.81       394
  soc.religion.christian       0.75      0.95      0.84       398
      talk.politics.guns       0.66      0.91      0.77       364
   talk.politics.mideast       0.72      0.97      0.83       376
      talk.politics.misc       0.65      0.60      0.63       310
      talk.religion.misc       0.92      0.26      0.41       251

             avg / total       0.79      0.77      0.76      7532

\end{alltt}

            \end{InvisibleVerbatim}
            
                \begin{InvisibleVerbatim}
                \vspace{-0.5\baselineskip}
    \begin{center}
    \includegraphics[max size={\textwidth}{\textheight}]{Naive_Bayes_files/Naive_Bayes_24_1.png}
    \par
    \end{center}
    
            \end{InvisibleVerbatim}
            
        
    


    % Make sure that atleast 4 lines are below the HR
    \needspace{4\baselineskip}

    
        \vspace{6pt}
        \makebox[0.1\linewidth]{\smaller\hfill\tt\color{nbframe-in-prompt}In\hspace{4pt}{[}83{]}:\hspace{4pt}}\\*
        \vspace{-2.65\baselineskip}
        \begin{ColorVerbatim}
            \vspace{-0.7\baselineskip}
            \begin{Verbatim}[commandchars=\\\{\}]
\PY{n}{phat\PYZus{}c}\PY{p}{,}\PY{n}{phat\PYZus{}w}\PY{p}{,} \PY{n}{pred}\PY{p}{,} \PY{n}{rep} \PY{o}{=} \PY{n}{bayes}\PY{o}{.}\PY{n}{run\PYZus{}model}\PY{p}{(}\PY{n}{twenty\PYZus{}train}\PY{p}{,} \PY{n}{twenty\PYZus{}test}\PY{p}{,}
                \PY{n}{beta}\PY{o}{=}\PY{l+m+mf}{0.1}\PY{p}{,}
                \PY{n}{bow}\PY{o}{=}\PY{p}{(}\PY{n}{train\PYZus{}tfidf}\PY{p}{,} \PY{n}{test\PYZus{}tfidf}\PY{p}{)}\PY{p}{,}
                \PY{n}{report}\PY{o}{=}\PY{n+nb+bp}{True}\PY{p}{)}
\end{Verbatim}

            
                \vspace{-0.2\baselineskip}
            
        \end{ColorVerbatim}
    

    

        % If the first block is an image, minipage the image.  Else
        % request a certain amount of space for the input text.
        \needspace{4\baselineskip}
        
        

            % Add document contents.
            
                \begin{InvisibleVerbatim}
                \vspace{-0.5\baselineskip}
\begin{alltt}estimating class priors
estimating word priors
estimating params with Dirchlet Prior:
         vocabsize = 11314.000000
         beta = 0.100000
         alpha = 1.100000
         denominator = 1131.400000
predicting
                          precision    recall  f1-score   support

             alt.atheism       0.83      0.70      0.76       319
           comp.graphics       0.76      0.74      0.75       389
 comp.os.ms-windows.misc       0.76      0.68      0.72       394
comp.sys.ibm.pc.hardware       0.67      0.74      0.70       392
   comp.sys.mac.hardware       0.88      0.82      0.85       385
          comp.windows.x       0.86      0.80      0.83       395
            misc.forsale       0.85      0.78      0.82       390
               rec.autos       0.88      0.91      0.89       396
         rec.motorcycles       0.91      0.95      0.93       398
      rec.sport.baseball       0.95      0.94      0.94       397
        rec.sport.hockey       0.93      0.98      0.95       399
               sci.crypt       0.78      0.95      0.86       396
         sci.electronics       0.78      0.72      0.75       393
                 sci.med       0.89      0.83      0.86       396
               sci.space       0.83      0.93      0.88       394
  soc.religion.christian       0.68      0.96      0.79       398
      talk.politics.guns       0.68      0.95      0.79       364
   talk.politics.mideast       0.93      0.96      0.95       376
      talk.politics.misc       0.92      0.55      0.69       310
      talk.religion.misc       0.92      0.33      0.48       251

             avg / total       0.83      0.82      0.82      7532

\end{alltt}

            \end{InvisibleVerbatim}
            
                \begin{InvisibleVerbatim}
                \vspace{-0.5\baselineskip}
    \begin{center}
    \includegraphics[max size={\textwidth}{\textheight}]{Naive_Bayes_files/Naive_Bayes_25_1.png}
    \par
    \end{center}
    
            \end{InvisibleVerbatim}
            
        
    
We improve our overall accuray a bit with the stronger TFIDF model,
taking the frequency of the words in the document into account and not
just the counts of the words in a document. This reduces some of the
bias in a long doument. Laplacian smoothing doesn't seem to change our
results very much in the original bow model.
        

        \renewcommand{\indexname}{Index}
        \printindex

    % End of document
    \end{document}


